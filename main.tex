\documentclass{article}

\usepackage{graphicx} % Required for inserting images
\usepackage{adjustbox}
\usepackage{amsmath}
\usepackage{ragged2e}
\usepackage{blindtext}
\usepackage{multicol}
\usepackage[left=15mm, top=17mm, right=16mm, bottom=18mm]{geometry}
\usepackage{enumitem}
\usepackage{parskip}
\usepackage[russian]{babel}
\linespread{0.8}
\setcounter{page}{105}




\begin{document}




\begin{figure}[h]
    
    \centering
    \includegraphics[width= 18cm]{Снимок экрана_24-9-2024_0273_proc.ostis.net.jpeg}
    \caption {Complexity of computer vision tasks. Object level tasks}
    \caption {Table I
Some ML/CV frameworks and libraries}
    \includegraphics[width= \textwidth]{Снимок экрана_26-9-2024_233326_proc.ostis.net.jpeg}
    \
    \begin{multicols}{2} 
\includegraphics[scale=0.6]{Снимок экрана_27-9-2024_02220_proc.ostis.net.jpeg}
\hspace{49mm}%
    \label{fig:enter-label} 
 
\setlength{\hsize}{0.9\hsize}% emphasize effects




 When selecting a neural network architecture to solve     
a given problem, there are several options. As a rule,
while working on projects, researchers and developers
apply to each of them in the sequence in which they will
be listed:
\begin{itemize}[noitemsep]
    \item 

 state-of-the-art (SOTA) architecture for specific
computer vision problems (classification, detection,
    segmentation, etc.);
    \columnbreak
    
\item modified state-of-the-art architecture for specific
computer vision problems (classification, detection,
segmentation, etc.);


\item costume state-of-the-art architecture for specific
computer vision problems (classification, detection,
segmentation, etc.).

 Some state-of-the-art architectures for computer vision
tasks are presented in Table II. This is by no means a
complete list, but it will provide insight into the variety of
neural network architectures for solving computer vision
problems.

The architecture selection process should also be
guided by the model size and its inference time. There
may be situations when it is necessary to develop a
lightweight model, for example, for cutting-edge devices,
and to provide a high speed of inference.

Another necessary component for the development of
computer vision solutions is computational resources.
It is possible to use both GPUs on workstations and
cloud computing resources (for example, Amazon Web
Services, Google Cloud Platform, Microsoft Azure, etc.).





\end{itemize}



\end{multicols}
\end{figure}
\newpage
\begin{figure}[h]
    \centering
    \includegraphics[width=\linewidth]{Снимок экрана_28-9-2024_175051_proc.ostis.net.jpeg}
    \label{fig:enter-label}
    \begin{multicols}{2}
    \textbf{NVIDIA, the market leader, offers deep-learning GPUs.}


    \textbf{IV. Data Importance}
        
    
        




The main blockers in machine learning projects are
data unavailability, an insignificant number of data, and
low variability. Often, the choice of neural network
architecture is not as critical as an underpowered dataset.
Also, data collection and labeling for model building can
take a significant amount of time. Especially recently,
much attention has also been paid to personal data, the
impossibility of using it, and data ethics in general.
Formulate some rules about data for ML model development:
\setlength{\parskip}{0pt}
\begin{itemize}[noitemsep]
    \item 
\par
 at a bare minimum, collect around 1000 examples.
   \item 
 for most "average" problems, collect 10,000 -
100,000 examples.
   \item 
 for “hard” problems like machine translation, high
dimensional data generation, or anything requiring
deep learning, collect 100,000 — 1,000,000 examples.
\end{itemize}
Generally, the more dimensions your data has, the
more data you need. It is necessary to have roughly 10
times the amount of data in your examples. The more
complex the problem, the more data you need.
A. Multiplicity of computer vision tasks and interdisciplinarity
We would like to emphasize that there is often a need
for special, interdisciplinary knowledge when solving
computer vision problems. This is evident when working
with a class of medical images. In such cases, it is necessary to involve medical experts both for data labeling
and for result interpretation. A good rule of thumb is to
engage several experts at the same time and average their
labels and estimates.
On the other hand, different computer vision problems
are possible for the same class of images. It is very
important to formulate the problem in terms of machine learning and computer vision (image classification,
object detection, image segmentation, etc.) at the very
beginning of the project. This will immediately give an
understanding of how to label images (polygon, class
label for a whole image, using bounding boxes, etc.)
and with what architectures and algorithms to perform
experiments (Table II).
\columnbreak
 As an example, here are images and possible computer
vision tasks. This is a class of medical images. Figure 5
shows examples of retina images and their labels for
classifying the stages of diabetic retinopathy. Figure 6
shows examples of optical disk detection. And Figure 7
shows the results of vessel segmentation in retinal images
[9]–[11].

\includegraphics[width=\linewidth]{Снимок экрана_30-9-2024_155825_proc.ostis.net.jpeg}


   
\paragraph{V. Methodology of machine learning model
development for Computer Vision
}


Summarize the above information and formulate a
methodology for machine learning model development
for solving computer vision tasks. There are three logical
levels to this methodology:
\begin{itemize}
    \item 
coding.
\end{itemize}




\end{multicols} 
\end{figure}

\newpage
\begin{multicols}{2}


    \includegraphics[width=\linewidth]{Снимок экрана_30-9-2024_16107_proc.ostis.net.jpeg}
\begin{itemize}[noitemsep]
    \item  model development.
\item working with data.
\end{itemize}
The methodology includes next stages:
\begin{itemize}[noitemsep]
\item Model Building. The computer vision task is formulated (classification, detection, segmentation, etc.),
the stack of technologies used is determined, data
is collected and labeled, promising model architectures are determined, and models are built. A common practice is to split the data set into a training
set, a validation set, and a test set. The training
dataset is used for model building. A validation
dataset is necessary to improve the training process.
\item Model Evaluation and Experimentation. The accuracy of the models is assessed on the test dataset
based on metrics for specific tasks. The best model
is chosen.
\item Productionize Model. The model is saved using the
selected format (ONNX, h5, etc.).
\item Testing. Code and model are tested using training
dataset.
\item Deployment. The deployment of the model in
the product environment is determined and implemented. There are several possible options:
– batch deployment;
– real time;
– streaming deployment;
– edge deployment.
\item Monitoring and Observability. Continuously monitoring and testing the model’s effectiveness p

  \end{itemize}







 deployment pose ongoing challenges. Regular monitoring is necessary to ensure accurate results, identify potential issues, and drive performance enhan
   
   
 
\textbf{VI. Challenges of building intelligent systems for
computer vision tasks}

Despite the great theoretical and technical progress
in the field of computer vision, there are a number of
limitations that need to be overcome in the future. Let’s
name some of them.
\begin{itemize}[noitemsep]
\item  The diversity of visual representation, such as illumination, perspective, or occlusion in objects, is a
major challenge. These variations must be overcome
to eliminate any visual inconsistencies.
\item  With each image consisting of millions of pixels,
dimensional complexity becomes another barrier to
overcome. This could be done by using different
techniques and methodologies.
\item  Real-time processing can be challenging. This
comes into play when making decisions for autonomous navigation or interactive augmented realities, which require optimal performance of computational frameworks and algorithms for fast and
accurate analysis.
\item  Ethical considerations are paramount in artificial
intelligence, and computer vision is no different.
This could be bias in deep learning models or any
discriminatory results. This emphasizes the need for
a proper approach to dataset curation or algorithm
development.
\end{itemize}
Outline promising directions in the field of computer
vision in the next few years: zero-shot learning, fewshot learning, and one-shot learning. It makes sense to
develop scientific research in this direction. Zero-shot
learning, few-shot learning, and one-shot learning are all
techniques that allow a machine learning model to make
predictions for new classes with limited labeled data. The
choice of technique depends on the specific problem and
the amount of labeled data available for new categories
or labels (classes) [12].
If we move away from technical blockers and think
about high-level analysis of computer vision challenges,
then it is obvious that the component design of hybrid
and intelligent systems will be promising in the future. It
is necessary to solve the problem of the compatibility of
scientific research results in the field of artificial intelligence. This problem is currently the key one preventing
the active development of artificial intelligence.
a significant reduction in the effectiveness of using the
component method. designing computer systems based
on reusable libraries components.
Insufficiently high degree of learning about modern
computer systems during their operation, which results

\end{multicols}

\end{document}
